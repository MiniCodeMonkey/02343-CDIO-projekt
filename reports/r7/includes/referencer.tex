%%%%%%%%%%%%%%%%%%%%%% Referencer %%%%%%%%%%%%%%%%%%%%%%
\usepackage{url} %til referencerne. sætter det i en verbatim-agtig font

\usepackage{varioref} % skal bruges til \vref
\usepackage{prettyref} % skal bruges til \prettyref. underline'r referencer og gør dem klikbare
\newcommand*{\pref}{\prettyref} % man er vel doven?
%\usepackage{mfirstuc}
%\newcommand*{\Pref}[1]{\xxmakefirstuc{\prettyref{#1}}}
% følgende retter vref-kommandoen til danske ord:
\newrefformat{cha}{kapitel~\vref{#1}}
\newrefformat{sec}{afsnit~\vref{#1}}
\newrefformat{fig}{figur~\vref{#1}}
\newrefformat{eq}{formel~\vref{#1}}
\newrefformat{tab}{tabel~\vref{#1}}
\newrefformat{app}{appendiks~\vref{#1}}
% Farverne på links

\usepackage{titletoc}
\usepackage[
    colorlinks%
%    ,linkcolor={darkgreend}%
    ,linkcolor={DTU9}%
    ,citecolor={DTU8}%
    ,urlcolor={DTU7}%
    %,plainpages=false%
    %,pdfpagelabels%
    %,bookmarks
    ,breaklinks=true%Tillad at dele bl.a. links i TOC over flere linjer
    ]{hyperref}% angiver hvilke farver links har. samme farve til alle links er at foretrække.
    
%\titlecontents{section}[1.5em]{}{\contentslabel{2.3em}}{\hspace*{-2.3em}}{\titlerule*[1pc]{.}\contentspage}
\contentsmargin{2em}
\dottedcontents{chapter}[1.5em]{\vspace{-0.4em}\Large\bf}{1em}{1pc}
\dottedcontents{section}[3em]{\vspace{-0.5em}\large\bf}{1.5em}{1pc}
\dottedcontents{subsection}[5em]{\vspace{-0.6em}\normalsize}{2.2em}{1pc}
\dottedcontents{figure}[3em]{\vspace{-0.9em}}{1.5em}{1pc}
\dottedcontents{table}[3em]{\vspace{-0.9em}}{1.5em}{1pc}
%\setlength{\cftbeforesecskip}{0.5ex}

%\usepackage{minitoc} % Ekstra TOC med forskellig dybde
%%%%%%%%%%%%%%%%%%%%%%%%%%%%%%%%%%%%%%%%%%%%%%%%%%%%%%%%
