\section{Gruppekontrakt}
\subsection{Gruppens medlemmer}
\begin{description}
	\item[JK] Jeppe Kronborg, s070162
	\item[PC] Per Boye Clausen, s042067
	\item[TB] Terkel Brix, s093482
	\item[MA] Morten Hulvej Andersen, s083117
	\item[MH] Mathias Hansen, s093478
\end{description}

\subsection{Kommunikation}
Korte/vigtige beskeder sendes gennem CampusNet gruppe som høj prioritet. Denne gruppe skal hos alle være indstillet til at sende høj-prioritet beskeder som SMS.

Løbende udvikling rapporteres gennem versionsstyringens push-notices.

Generelt udveksles løbende information gennem mødes; hver planlagt fælles-aktivitet begyndes med startmøde, og afsluttes med gå-hjem møde.

Afbud ifm. fælles aktivitet meldes hurtigst muligt som højprioritet besked på CampusNet -- alternativt direkte til projektleder -- senest ved mødestart. For sent afbud noteres som fravær med note.

\subsection{Møder}
I 13-ugers prioden er mødetiden som udgangspunkt hver onsdag kl. 8.15-12, med start-møde kl. 10.\\
Øvrige arbejdstider aftales løbende, og hvert onsdags-startmøde tager stilling til individuel indsats.

Ved hvert møde udarbejdes en mødelog ud fra skabelon.

\subsection{Roller}
\begin{center}
\begin{tabular}{l l}
	Projektleder:		& PC \\
	Stedfortræder:		& MA \\
	Materialeansvarlig:	& JK \\
	Dokument-ansvarlig:	& MA
\end{tabular}
\end{center}

\subsection{Dispositioner}
Rapporter udarbejdes i \LaTeX. Til programmering er overordnet valgt Java.

\subsection{Dokumenthåndtering}
\begin{center}
\begin{tabular}{l l}
	Mødelogs:			& Dropbox \\
	Kildekode:			& git \\
	Rapport:			& git \\
	Diverse dokumenter:	& Dropbox
\end{tabular}
\end{center}
