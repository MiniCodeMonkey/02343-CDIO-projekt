\chapter{Indledning}

\section{Kravspecifikationer}
Der skal, vha nedenstående udstyr, udvikles en robot og et styresystem, der autonomt kan navigerer i en labyrint, undgå forhindringer, opsamle og transportere objekter ud.
Materialerne til rådighed er:
\begin{itemize}
\item 2 Lego NXT robotsæt\footnote{Educational Version, LEGO \#9797} inkl. opladeligt batteri,batterioplader, bluetooth USB-dongle, NXT brick, diverse Lego-klodser, diverse sensorer.
\item 2 Lego NXT resourcesæt\footnote{LEGO \#9648} med diverse lego klodser
\item 1 webcam
\item 1 stativ til montering af webcam
\item Flamingoplader til opbygning af banen
\end{itemize}

Labyrinten består af en antal identiske hvide forhindringer, der måler 30x30x5 cm. Ligeledes indeholder labyrinten et antal identiske røde kubiske objekter, der kaldes "kager". Labyrinten er rektangulær og hjørnerne udgøres af forhindringer. Der kan placeres et arbitræt antal forhindringer indenfor labyrintes grænser, disse skal dog placeres med minimum 30 cm mellemrum, med mindre de berører hinanden. Labyrintens størelse kan maximalt være 180x240 cm

\section{Konkurerencen}
På konkurrencedagen vil projektet blive vurderet udfra hvordan robotten klarer sig i labyrinten. Robotterne har 10 minutter til at køre alle kagerne ud. Der vil blive givet positive point når en kage flyttes og køres ud af banen, og positive point for hvor meget tid der er tilbage. Negative point vil blive givet hvis robotterne kører ind i en forhindring, eller en anden robot.