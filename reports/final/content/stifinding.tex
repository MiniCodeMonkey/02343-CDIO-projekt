\chapter{Implementering}

\section{Valg af algoritme}
Efter nøje research af mulige algoritmer er valget faldet på \textbf{A*} algoritmen.
Algoritmen er valgt på baggrund af tidligere erfaringer fra nogle af gruppens medlemmer samt fordi at den opfylder projektets behov i forhold til effektivitet og brugbarhed. Desuden er algoritmen meget udbredt og har vist sit værd i uttalige software projekter tidligere.

\textbf{Dijkstra's algoritme} har også været overvejet, da denne algoritme kunne bruges som base for en videreudvikling i forhold til vores krav, dette blev dog droppet da det ikke regnedes for, ikke at kunne betale sig.

\section{Implementering}
Stifindingen blev implementeret som en seperat pakke i projektet med dertilhørende klasser der henholdsvis repræsenterer \texttt{TileMap}, \texttt{Path} og \texttt{Step}s. Implementeringen er baseret på en eksisterende implementeringen af \textit{Kevin Glass}.
% Kilde: http://www.cokeandcode.com/pathfinding

Dertil er implementeringen blevet optimeret og justeret til projektets behov.

\section{Optimering}
Algoritmen er optimeret således at antallet af steps der returneres af \texttt{findPath(...)} metoden er reduceret til et minimum, dette er gjort for at behjælpe controlleren i arbejdet med at instruerer robottens bevægelser. Antallet af steps er reduceret ved at loope igennem alle steps, og for hvert step benytte \textit{afstandsformlen} og kun tage step'et med i path'en såfremt afstanden til det forrige valgte step er over en foruddefineret grænseværdi.