\chapter{Analyse}

\section{Krav og mål}
\subsection{Succeskriterier}
\begin{enumerate}
	\item Aflevering af alle afleveringer rettidigt
	\item 1 robot konstrueret med Lego Mindstorms NXT, kan modtage kommandoer via Bluetooth.
	\begin{enumerate}
		\item Robotten skal kunne opsamle 'kager' -- røde terninger -- og flytte disse.
	\end{enumerate}
	\item PC-applikation skal styre opsamling og flytning af 'kagerne':
	\begin{enumerate}
		\item Billedtagning og -behandling
		\begin{enumerate}
%			\item Hent billeder fra USB webcam
%			\item Behandl billeder og identificér bane, robot, 'kage' og forhindringer
			\item Bestem position og retning for robot
			\item Bestem positioner for kager
			\item Bestem positioner for forhindringer
		\end{enumerate}
		\item Stifindingsalgoritme
		\begin{enumerate}
			\item Bestem rute fra robot til kage -- uden om forhindringer -- ud fra positioner
		\end{enumerate}
		\item Kommunikation med robot
		\begin{enumerate}
			\item Dirigér robot ad bestemt rute vha. Bluetooth kommunikation
		\end{enumerate}
		\item Styring (vha. ovenstående)
		\begin{enumerate}
			\item Lokalisér kager, robot og forhindringer på banen
			\item Dirigér robot til én kage -- uden om forhindringer -- og saml kagen op
			\item Dirigér robot uden for banen -- stadig uden om forhindringer -- og læg kagen
			\item Gentag, så længe der er kager på banen
		\end{enumerate}
	\end{enumerate}
\end{enumerate}

\subsection{Mål}
\begin{tabular}{r p{0.85\linewidth}}
	\textbf{Prio} & \textbf{Mål} \\
	\hline
%	1 & Flere kager, som alle skal fjernes fra banen af robotten.\\ % Succeskriterie
	1 & 2 robotter til at udføre opgaven i fællesskab. Robotterne må ikke kollidere eller gå efter samme "kage".\\
	2 & Fejlhåndtering -- sikkerhed for at undgå forhindringer, kalibrere billedbehandling mv.\\
	3 & Mulighed for at håndtere kager og/eller forhindringer, der bliver flyttet. \\
	$\bullet$ & 1. plads i konkurrencen
\end{tabular}

\section{Løsningsstrategi}
Fra starten af, er der defineret en klar opdelig af projektets hoveddele. Herunder er hver af disse hoveddeles løsningsstrategi beskrevet.

\subsection{Robot}
Det vælges fra begyndelsen at benytte projektbeskrivelsens anbefalinger i henhold til robottens features og interface med computer.
Dette betyder også at der benyttes \textit{Lejos} som NXT'ens styresystem og at der vælges bluetooth til styring af robotten.
Robotten forsøges iterativt udviklet til et endeligt produkt.

\subsection{Billedbehandling}
Der arbejdes ud fra et simpelt udgangspunkt. Så meget som muligt opbygges fra grunden for at give mulighed for ekstra optimering.

Der vil løbende overvejes, om ekstra funktionalitet skal tilføjes, eller om der er behov for ekstra optimering.

\subsection{Stifinding}
Der arbejdes efter at finde en eksisterende stifindings algoritme som base for stifindingen til projektet. Algoritmen bør derefter tilpasses til robot styringsdelens behov.
Det er vigtigt at køretiden på stifindingen er så minimal som muligt således at der konstant kan beregnes en sti for robotten for at tage højde for eventuelle dynamiske ændringer i banen eller dens objekter.