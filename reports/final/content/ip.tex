\chapter{Billedbehandling}\label{cha:ip}
Billedbehandlingen er udviklet helt fra grunden. Der er valgt en metode, som så tidligt som muligt fortolker billedet fra webcam til et 2D-array. Dette er gjort for at få maksimalt kendskab til koden, og dermed bedre mulighed for at rette eventuelle uhensigtsmæssigheder samt forbedre performance.

Billedbehandlingen er udviklet iterativt. I projektets forløb har ønskerne til billedbehandlingen ændret sig, og i første omgang er disse ønsker opfyldt hurtigst muligt for at maksimere effektiviteten i projektforløbet.

Efter første full cycle er erfaringerne og tilføjelserne i billedbehandlingen benyttet til udvikling af en ny version -- version 2. Her er køretid mv. forsøgt optimeret i forhold til tidligere.

\section{Design}
Billedbehandlingen inddeles i to dele: Webcam og behandling af billedet.

\subsection{Webcam}
Webcam-delen sørger for at varetage forbindelsen til webcam, og herfra hentes rå billeder.

Der defineres et interface, \texttt{IImageSource}. Dette specificerer metoderne \texttt{init()}, som benyttes til at forbinde til webcam -- \texttt{close()}, som lukker forbindelsen til webcam -- samt \texttt{getImage()}, som returnerer et billede fra webcam som et \texttt{BufferedImage} objekt.

Til at håndtere selve forbindelsen til kameraet benyttes JMF \rec{cite}.

\subsection{Billedbehandling}
Billedbehandlingen behandler billedet fra kameraet og bestemmer positioner af forhindringer, kager og robotter, nærmere bestemt:
\begin{description}
	\item[Fortolkning af kildebillede] til et \textit{tilemap\footnote{Tilemap er et 2D-array af integers, som er opbygget som et billede med meget få farver (én farve pr. genkendt objekttype)}}. Hver pixel undersøges i forhold til fastsatte grænseværdier.
	\item[Filtrering] af tilemap, hvor områder af pixels sorteres fra, hvis ikke de dækker et tilstrækkelig stort antal sammenhængende pixels. Dette fjerner støj fra billedet, og sikrer mod fejlagtig genkendelse af objekter.
	\item[Bestemmelse af grænser] for banen ud fra de 4 hjørne-forhindringer.
	\item[Generering af map af forhindringer], hvor der er tilføjet buffer-zoner omkring forhindringer. Hvis det ønskes, kan robot 2 her markeres, ligeledes med en buffer-zone, i tilfælde, hvor robot 1 skal finde vej uden om.
	\item[Bestemmelse af position] for kager.
	\item[Bestemmelse af position og vinkel] for robotter.
	\item[Skalering af output] for at optimere køretiden for stifindingen. Denne funktionalitet er ikke taget i brug.
	\item[Generering af grafisk repræsentation] af de behandlede data, så det er muligt at følge billedbehandlingens arbejde løbende.
\end{description}

\subsubsection{Opbygning}
Der er specificeret et interface -- \texttt{IImageProcessor} -- som billedbehandlingen skal implementere. I dette interface lægges også standard-værdier for mange af de parametre, som billedbehandlingen gør brug af.

Selve billedbehandlingen er implementeret i \texttt{ImageProcessor2} klassen. \rec{Forklaring på 2?} Al funktionalitet er specificeret her.

Data, som skal benyttes videre i det samlede system, returneres i DTO\footnote{Data Transfer Object} klassen \texttt{Locations}, som implementerer \texttt{ILocations} interfacet. Disse entiteter er rent databærende.\\
\texttt{Locations} gemmer tilemap og forhindrings map som 2D int-arrays, og kager og robotter gemmes som lister af hhv. \texttt{Cake}- og \texttt{Robot} DTO-objekter. Kildebillede og fortolket billedet gemmes som \texttt{BufferedImage} objekter.

Der er til de fleste parametre lavet getter/setter metoder. Til bestemmelse af grænseværdier for en objekttype er indført en \texttt{Thresholds} klasse, som indeholder minimums- og maksimumsværdier for farvekomponenterne.

\section{Implementering}
\subsection{Webcam}
Der benyttes fortrinsvis JMF til hele implementeringen af webcam forbindelsen. Implementeringen tager udgangspunkt i det eksempel, som er givet på CampusNet. \rec{Ref?}\\
I \texttt{init()} metoden oprettes der forbindelse til enheden "`vfw:Microsoft WDM Image Capture (Win32):0"'. Der benyttes formatet 320x240 RGB.

Selve forbindelsen bruges gennem et statisk \texttt{Player} objekt.

\subsection{Billedbehandling}
Billedbehandlingen benyttes gennem metoden \texttt{examineImage(sourceImage, debug)}, som returnerer et \texttt{Locations} objekt, indeholdende:
\begin{description}
	\item[Et \textit{tilemap}], hvori alle fundne objekter er repræsenteret på pixel-niveau.
	\item[Et \textit{obstaclemap}], som er tilsvarende, blot for forhindringer samt en bufferzone omkring disse.
	\item[En liste af \textit{kager}], som er en \texttt{ArrayList} af \texttt{Cake} objekter, som hver holder positionen for en kage.
	\item[En liste af \textit{robotter}], som er en \texttt{ArrayList} af \texttt{Robot} objekter, som hver holder position og vinkel i radianer for en robot. Her returneres altid 2 robotter i samme rækkefølge, også, hvis en robot ikke kan ses på billedet.
	\item[\textit{Kildebilledet}], som er behandlet (såfremt \texttt{debug} er sandt).
	\item[Et billede], som viser en grafisk fortolkning af de fundne objekter mv.
\end{description}

Billedbehandlingen i \texttt{examineImage(...)} følger proceduren
\begin{enumerate}
	\item Fortolk kildebillede og generér tilemap.
	\item Filtrér for små forhindringer fra for at undgå støj.
	\item Find grænser for selve banen og gem disse.
	\item Registrér kager og robotter -- og filtrér herunder for små objekter fra -- og sæt bufferzone omkring forhindringer i obstaclemap.
	\item Generér \texttt{Locations} objekt.
	\item \textit{Hvis \texttt{debug} er sandt,} gem kildebillede, generér og gem grafisk fortolkning i \texttt{Locations} objektet.
	\item Returnér \texttt{Locations} objektet.
\end{enumerate}

Der er indført optimeringer for at forbedre køretiden. Navnligt behandles kun hver 3. pixel på hver 3. linje ved registrering af objekter. Dette er gjort, at denne proces er den klart mest ressourcekrævende. Objekter registreres stadig ned til enkelte pixels, og idet små objekter alligevel filtreres fra, vil der kun i ekstreme tilfælde mange objekter i registreringen.

Mere dybdegående beskrivelse af implementeringen af billedbehandlingen kan ses i \pref{app:ip-impl}.

\section{Test}
Mere uddybende omtale af test kan ses i \pref{app:ip-test}.
\subsection{Billedbehandling}
\paragraph{Funktionalitet}
Funktionstest af billedbehandlingen er gennemført ved visuel inspektion. Et simpelt testprogram er udviklet til at forbinde til en billedkilde og fortolke billedet fra denne. Kildebilledet vises så med fortolkningen under. De tre knapper fra webcam testen går igen.

Der er ved test af billedbehandlingen benyttet både genererede billeder samt "`rigtige"' billeder fra webcam til at vurdere, hvorvidt billedbehandlingen fungerer som ønsket. Samtidig er der gennem konsol-output aflæst vinkler på robotter, som vurderes op imod de reelle vinkler.

\paragraph{Performance}
Der er gennemført benchmark test af billedbehandlingen i flere udgaver. Dette har vist, hvorledes den seneste udgave af billedbehandlingen er klart hurtigere end de tidligere. Målingerne ved disse test kan ses i tabellen i \pref{app:ip-test}.

\begin{comment}
	
Webcam
	Varetager forbindelse til webcam
	Henter BufferedImage i 320x240

Processor
	Billede ind som BufferedImage
	Parse til 2D-array
	Filtrering
	Optimering i walker
	Objekt-/robot ID
	Obstacle map
	Grafisk repræsentation
	Scaling (ej i brug)
	Returnerer Locations objekt til brug i pathfinder
	
Udviklingsproces
	Simpelt udgangspunkt
	Løbende tilføjet funktionalitet med behov fra stifinding
	Efter første full cycle udviklet i version 2 med optimeringer
	
Test
	Program t visuel test
	"Webcam-simulator"
	Benchmark kørsler, resultater
	Performance udvikling
\end{comment}