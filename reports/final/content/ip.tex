\chapter{Billedbehandling}\label{cha:ip}
\section{Design}
Billedbehandlingen inddeles i to dele: Webcam og behandling af billedet.

\subsection{Webcam}
Webcam-delen sørger for at varetage forbindelsen til webcam, og herfra hentes rå billeder.

Der defineres et interface, \texttt{IImageSource}, som specificerer metoderne \texttt{init()}, som benyttes til at forbinde til webcam -- \texttt{close()}, som lukker forbindelsen til webcam -- samt \texttt{getImage()}, som returnerer et billede fra webcam som et \texttt{BufferedImage} objekt.

Til at håndtere selve fobinddelsen til kameraet benyttes JMF \rec{cite}.

\begin{comment}
Opbygning
	Webcam -> behandling
	JMF
\end{comment}

\subsection{Billedbehandling}
Billedbehandlingen behandler billedet fra kameraet og bestemmer positioner af forhindringer, kager og robotter, nærmere bestemt:
\begin{description}
	\item[Fortolkning af kildebillede] til et \textit{tilemap\footnote{Tilemap er et 2D-array af integers, som er opbygget som et billede med meget få farver (én farve pr. genkendt objekttype)}}. Hver pixel undersøges i forhold til fastsatte grænseværdier.
	\item[Filtrering] af tilemap, hvor områder af pixels sorteres fra, hvis ikke de dækker et tilstrækkelig stort antal sammenhængende pixels. Dette fjerner støj fra billedet, og sikrer mod fejlagtig genkendelse af objekter.
	\item[Bestemmelse af grænser] for banen ud fra de 4 hjørne-forhindringer.
	\item[Generering af map af forhindringer], hvor der er tilføjet buffer-zoner omkring forhindringer. Hvis det ønskes, kan robot 2 her markeres, ligeledes med en buffer-zone, i tilfælde, hvor robot 1 skal finde vej uden om.
	\item[Bestemmelse af position] for kager.
	\item[Bestemmelse af position og vinkel] for robotter.
	\item[Skalering af output] for at optimere køretiden for stifindingen. Denne funktionalitet er ikke taget i brug.
	\item[Generering af grafisk repræsentation] af de behandlede data, så det er muligt at følge billedbehandlingens arbejde løbende.
\end{description}

\subsubsection{Opbygning}
Der er specificeret et interface -- \texttt{IImageProcessor} -- som billedbehandlingen skal implementere. I dette interface lægges også standard-værdier for mange af de parametre, som billedbehandlingen gør brug af.

Selve billedbehandlingen er implementeret i \texttt{ImageProcessor2} klassen. \rec{Forklaring på 2?} Al funktionalitet er specificeret her.

Data, som skal benyttes videre i det samlede system, returneres i DTO\footnote{Data Transfer Object} klassen \texttt{Locations}, som implementerer \texttt{ILocations} interfacet. Disse entiteter er rent databærende.\\
\texttt{ILocations} gemmer tilemap og forhindrings map som 2D int-arrays, og kager og robotter gemmes som lister af hhv. \texttt{Cake}- og \texttt{Robot} DTO-objekter. Kildebillede og fortolket billedet gemmes som \texttt{BufferedImage} objekter.

\section{Implementering}
\subsection{Webcam}
Der benyttes fortrinsvis JMF til hele implementeringen af webcam forbindelsen. Implementeringen tager udgangspunkt i det eksempel, som er givet på CampusNet. \rec{Ref?}\\
I \texttt{init()} metoden oprettes der forbindelse til enheden "`vfw:Microsoft WDM Image Capture (Win32):0"'. Der benyttes formatet 320x240 RGB.

Selve forbindelsen bruges gennem et statisk \texttt{Player} objekt.

\subsection{Billedbehandling}
I billedbehandlingen ligger klassevariable med tilhørende get-/set-metoder til de parametre, som er specificeret i \texttt{IImageProcessor}.

\paragraph{\texttt{examineImage(...)}}
Denne metode benyttes, når et billede skal behandles. Metoden tager som argumenter det kildebillede (et \texttt{BufferedImage}), som skal behandles, samt en boolsk værdi, som dikterer, hvorvidt en grafisk repræsentation af det behandlede billede skal dannes.\\
Metoden returnerer et \texttt{Locations} objekt.

\texttt{examineImage(...)} benytter de øvrige metoder i billedbehandlingen til at behandle det givne kildebillede.

\paragraph{\texttt{generateTilemap()}}
Her dannes ud fra kildebilledet et 2D-array med samme størrelse. Hver pixel undersøges i forhold til \rec{\texttt{Threshold}} objekter. Der tjekkes i rækkefølgen \textit{forhindring}, \textit{kage}, \textit{robot 1 (front-bag)}, \textit{robot2 (front-bag)}. Hvis ikke en pixel bliver genkendt her, tolkes den som værende baggrund/gulv.\\
Metoden gemmer det resulterende \textit{tilemap} i klasse-variablen \texttt{tilemap}.

\paragraph{\texttt{filterObstacles()}}
Filtrering af forhindrings-pixels foregår her. Samtidig foretages første generering af forhindrings map -- endnu et 2D-array, \texttt{obstaclemap}, med samme størrelse som \texttt{tilemap}.\\
Der benyttes endnu et 2D-array til at holde styr på behandlede pixels.

Metoden vandrer igennem alle pixels i \texttt{tilemap}. Hver gang en forhindring, som ikke allerede er behandlet, registreres, benyttes hjælpemetoden \rec{\texttt{collectRecursion()}} til at samle alle de sammenhængende forhindrings-pixels.\\
Hvis antallet af opsamlede koordinater er mindre end grænsen \texttt{MIN\_OBJECT\_SIZE}, forkastes forhindringen, og de fundne pixels defineres som baggrund i \texttt{tilemap} og \texttt{obstaclemap}.\\
I modsat fald -- hvis den opsamlede forhindring er tilstrækkelig stor -- registreres de opsamlede punkter i \texttt{obstaclemap}.

\paragraph{\texttt{findBounds()}}
Grænserne for selve banen bestemmes groft ved at finde den øverste og nederste vandrette linje i \texttt{tilemap}, hvor der er min. 5 forhindrings-pixels. Tilsvarende gælder for lodrette linjer mod venstre og højre.

De fundne grænseværdier gemmes som \texttt{int}-array på formen \textit{$\left\{\right.$ top, venstre, bund, højre $\left.\right\}$}.

\begin{comment}
Webcam
	mode
	player?
	
Processor
	Thresholds
	Resolution
\end{comment}

\section{Test}

\section{Udviklingsproces}

\begin{comment}
	
Webcam
	Varetager forbindelse til webcam
	Henter BufferedImage i 320x240

Processor
	Billede ind som BufferedImage
	Parse til 2D-array
	Filtrering
	Optimering i walker
	Objekt-/robot ID
	Obstacle map
	Grafisk repræsentation
	Scaling (ej i brug)
	Returnerer Locations objekt til brug i pathfinder
	
Udviklingsproces
	Simpelt udgangspunkt
	Løbende tilføjet funktionalitet med behov fra stifinding
	Efter første full cycle udviklet i version 2 med optimeringer
	
Test
	Program t visuel test
	"Webcam-simulator"
	Benchmark kørsler, resultater
	Performance udvikling
\end{comment}