\chapter{Videreudvikling}
\section{Robot}
\subsection{Sensorere}
Det vidste sig gennem udviklingen af robotterne, at der kunne opstå et problem, hvis robottens indgangsvinkel til kagen ikke var helt præcis, ville de blot gribe ved siden af kagen, og derefter gå i \texttt{DELIVERING}-state, og afleverer luft. En løsning der her blev diskuteret, var at sætte en farve-sensor over kloen, der derved kunne detektere om kagen rent faktisk var samlet op.

\subsection{Pilot Til Styring}
I slutningen af projektforløbet blev det klart at selvom robotten var præcis, havde vi nogle ydelsesmæssige problemer i forhold til tiden robotten brugte på at finde den rette vinkel. Dette ville kunne løses ved at implementere Pilot fra Lejos, der kan dreje robotten til en bestemt vinkel med en usikkerhed på +/- 10°. Vi kunne herefter kombinere det med det eksisterende software der drejer robotten ind på den helt rigtige vinkel.

\section{Billedbehandling}
\subsection{Autokalibrering Af Farver \& Lyssætning}
\rec{skal udfyldes}
\section{Samlet System}
\subsection{Vektor Beregning Af Vinkler}
\rec{skal udfyldes}