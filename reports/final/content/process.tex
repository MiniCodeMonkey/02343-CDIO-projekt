\chapter{Proces}\label{cha:process}
Der har været en del fokus på strukturen i gruppen, og der blev i begyndelsen af projektforløbet brugt en del tid på at få grundlagt gruppen samt nogle rutiner. Her blev også truffet nogle valg omkring valg af teknikker -- f.eks. strukturen af gruppen, medier til udveksling af beskeder, dokumentformater og fillokationer.

\section{Gruppestruktur}
Gruppen er baseret på en klar, overordnet fordeling af roller. Der er udpeget en projektleder (PC), en stedfortræder (MA), en materiale-ansvarlig (JK) samt en dokument-ansvarlig (MA). Disse roller definerer de overordnet ansvarlige og koordinerende roller, men alle medlemmer af gruppen har i projektets natur haft et medansvar på alle områder.

Der er også uddelegeret overordnede udviklingsmæssige ansvarsområder. Disse roller har skiftet noget undervejs efter behov.

Gruppestrukturen er valgt på baggrund af gode erfaringer fra tidligere projekter med tilsvarende gruppestørrelse og -sammensætning.

Der er udarbejdet en gruppekontrakt, hvori grundstrukturen og -reglerne er nedfældet. Gruppekontrakten er ikke nødvendigvis overholdt 100\% -- men har mest fungeret som et fælles standpunkt, som kan benyttes ved udredning af eventuelle problemer i gruppen.

Hver arbejdsdag\footnote{\textit{Arbejdsdag} dækker over en skemadag (onsdag formiddag) i 13-ugers perioden, arbejdsdag i 3-ugers perioden, eller fælles samling uden for skema} begyndes med et møde, hvor status, selvstændigt arbejde siden sidste møde, tidsforbrug samt program for dagens arbejde diskuteres. Arbejdsdagen sluttes også af med et møde, hvor fremskridt vurderes og eventuelle ekstra opgaver uddelegeres. Disse møder logføres til brug ved opgørelser over tidsforbrug, fremskridt mv.

\section{Forløb}
Projektperioden er, for det meste, forløbet planmæssigt. Der har været behov for at justere planer og mål nogle gange undervejs.

Ved 3-ugers periodens begyndelsen valgte gruppen at fokusere på udvikling frem for stabilitet ved at vælge at køre med 2 robotter. Dette har givet anledning til et væld af udfordringer, samt resulteret i et system, som er mindre stabilt end ellers. Dette kan til dels hænge sammen med, at flere af gruppens medlemmer sideløbende har været aktive på universitetets vegne omkring projekt-demonstrationer ved en international CDIO-konference. Dette har kostet noget tid og overskud i forhold til projektet og udfordringerne i at få 2 robotter til at fungere sammen.

Der er i gruppen enighed om, at samarbejdet har fungeret godt. Alle medlemmer har været engagerede og har gerne ydet en ekstra indsats, når det har været nødvendigt.\\
Største problematik har været, at MH har været bortrejst flere dage i 3-ugers perioden. Dette var annonceret i god tid, men er alligevel foregået på et uheldigt tidspunkt i forhold til projektet, og arbejdet har i denne periode ikke fungeret optimalt. Dette har dog umiddelbart vist sig at bero på nogle misforståelser, hvor der i gruppen har været forskellige opfattelser af, hvordan arbejdet skulle fortsætte i perioden\\
Detaljerne for rejseperioden burde have været diskuteret og udførligt noteret i god tid inden afrejsen, således at misforståelsen kunne være undgået.

\section{Kvalitet}
Der er benyttet en del \textit{prototyping}\footnote{Fokus på udvikling af ikke-komplette, fungerende komponenter} i udvikling. Dette har givet anledning til mange iterationer og mulighed for at fange mange fejl-antagelser tidligt i processen. Dette ses navnligt i robotten og billedbehandlingen.\\
Foruden at afdække fejl har fremgangsmåden hjulpet med til at finde ud af de helt konkrete krav, der stilles til komponenterne. Herved har det været muligt at benytte meget simple løsninger, frem for at bruge ressourcer på udvikling af avancerede løsninger, som kan vise sig at være overflødige -- eller måske endda ressourcekrævende at benytte.