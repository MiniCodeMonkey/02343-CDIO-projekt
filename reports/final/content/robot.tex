\chapter{Robot}

\section{Kommunikation}
Til at kommunikere med robotterne er det valgt at benytte \textit{Bluecove} biblioteket, som er en del af hele \textit{Lejos} biblioteket, hvilket derfor var oplagt at benytte. Desuden findes der rigt med eksempler online i forbindelse med brug af \textit{Bluecove} og \textit{Lejos} sammen.

\section{Styring}
Styringen foregår ved at der er lavet nogle metoder der sætter en eller begge motorer i en bestemt tilstand. De forskellige tilstande er at køre med en bestemt hastighed og at stoppe. Metoderne returnerer efter de er blevet kaldt, hvilket vil sige at en motor kører indtil at den bliver stoppet. Det er en meget primitiv måde at styre motorerne, til gengæld betyder det at der haves fuld kontrol over motorerne.
Dette har så betydet at det har været muligt at lave robotten meget præcis, da der ikke har været benyttet \textit{piloting}\footnote{Piloting er et system, indbygget i Lejos, som benytter feedback fra motorer til navigation} til at styre robotten.

\section{Iterationer}
Det blev i gruppen hurtigt besluttet at det indledende design af en simpel robot, havde meget høj prioritet, da det blev vurderet at det var vigtigt at have en fungerende robot at teste på. Og en videreudvikling derfor ville tage afsæt i dette design.
Det blev indledningsvis vurderet at larvefødder ville være den bedste fremgangsmåde, da disse ville have et større overfalde areal, og derfor være mindre afhængige af underslaget for at udgå hjulspin ved start og stop. Det var klart, allerede på dette tidspunkt, at denne design beslutning ville resultere i en langsommere robot, med en mere præcis navigering end hjul ville kunne give.
Det viste sig dog at et design med hjul leverede en tilfredsstillende præcision samtidig med en øget hastighed.

\billede{!htbp}{0.4}{BERTA-0_1}{Tidligere udgave af robot design}
\billede{!htbp}{0.4}{BERTA-1_0}{Slut udgave af robot design}

Gennem iterationerne skete der ligeledes en forbedring af selve konstruktionen efterhånden som designet blev forbedret. Gruppen har brugt enkelte medlemmers ekspertise gennem udviklingen af robotten, hvilket gav en betydelig forbedring af det overordnede design, herunder en bedre integration af - og tilgang til NXT blokken.

I den indledende fase af designet, blev der researchet en del på nettet, hvor et design til en klo blev fundet, som blev bibeholdt fra start til slut i projektet. Fordelen ved denne klo, er den at den kun drives af en enkelt motor. I andre design ville en motor lukke og åbne kloen, mens en anden ville løfte kloen. Valget af denne klo sparede derfor gruppen for at skulle benytte en ekstra NXT blok, da disse kun har 3 RJ-35 tilslutninger for motorer.

Efterhånden som konstruktionen blev gjort mere og mere og robust gennem diverse konstruktionsmæssige forbedringer, viste det sig at stivhed ikke nødvendigvis altid er den bedste fremgangsmåde.
I forbindelse med betjeningen af kloen, identificeredes der et problem der opstod når kloens position skulle bestemmes, dvs. om den fra starten var åbnet eller lukket. Da denne positionsbestemmelse blev mere mere og mere afgørende efterhånden som konstruktionen blev mere stiv og rigid, kunne det ses at motoren i værste fald ville have mulighed for at rive konstruktionen fra hinanden. Dette blev løst ved at designet i kraftoverførslen mellem klo-motoren og selve kloen blev gjort løs som betydede at en "`overdrejning"' ikke ville have nogen betydning. Og som en bonus ville det så være muligt, med sikkerhed, altid at kunne bestemme kloens position ved blot at "`overdreje"' den i 2 sekunder før starten af hver test.

\section{Fra 1 til 2 robotter}
Beslutningen om at arbejde mod at benytte 2 robotter blev truffet meget tidligt i forløbet. Den endelige beslutning blev først truffet umiddelbart efter det sidste styregruppemøde. Det var netop også her at styregruppen pointerede, at der skulle tages en beslutning om der skulle satses på at lave en enkelt robot og optimere den, eller bruge to. Valget af to robotter blev taget ud fra begrundelsen, og forventningen om at to robotter der arbejdede sammen ville kunne rydde banen hurtigere. Kravene til kommunikation og robotternes indbyrdes forhold ændrede sig dog drastisk. Både fordi kommunikationen var i en Singleton-klasse, og fordi det skulle undgås at robotterne tog de samme kager, og kørte ind i hinanden.

\section{RMI}
Kommunikationen til en robot bliver oprettet gennem en Singleton-klasse, problemet løses ved at kommunikationen til hver robot får sin egen process. Det er således valgt at benytte IPC mellem processerne via RMI.

\billede{!htbp}{0.4}{bluetooth-connection-overview}{Oversigt over NXT flow}

På figur~\vref{fig:bluetooth-connection-overview} ses hvordan programmet har byttet om på det traditionelle klient-server forhold. Det er gjort for at sikre at programmet og NXT'ernes Bluetooth forbindelse kører i seperate processer. Programmet starter en ny process og giver NXT'ens MAC-addresse med som argument. Processen virker som en RMI server, og starter en forbindelse til NXT'en, og lader derefter RMI-klienten kalde metoder fra et interfacet \texttt{IControl} så det er muligt at tilgå de metoder der starter og stopper motorerne på NXT'erne.

RMI er den oplagte løsning da kommunikationen til de andre processor foregår transparent som tilsyneladende normale metode kald. På den måde kaldes metoderne processerne så effektivt som muligt på en distribueret måde.

\section{Robotternes roller}
For at undgå kollisioner og at robotterne går efter de samme kager, har de fået nogle indbyrdes roller; én er master, og én er slave. Masteren har forkørselsret hvis robotterne kommer for tæt på hinanden, og slave-robotten sættes i en yielding-state, hvis den kommer for tæt på masteren. Master-robotten kan derefter navigere uden om slave-robotten når denne er i en yielding-state.

Robotten vælger altid den kage der er nærmest, og som ikke allerede er valgt og robotten begiver sig herefter hen efter den. Når der kun er 2 kager tilbage er de reserveret til masteren. Det er gjort for at undgå at slave-robotten kommer til at køre med den sidste kage og komme for tæt på master-robotten. Den ville således gå i en yielding state, og da der ikke er flere kager ville master-robotten stå og vente til der kommer nye kager, hvilket vil resultere i at slave robotten aldrig får afleveret kagen.

\section{Test}
Til at teste de metoder som er blevet lavet til at styre robotten har gruppen lavet et testtool til manuel styring af motorerne. Dette har givet god viden om hvordan motorerne virker mekanisk, hvordan robotten styres og hvordan der kommunikeres med robotten. Denne viden er blevet brugt i de metoder der er lavet i interfacet som deles af RMI. Desuden har det været forsøgt at få implementeret en \textit{slowstart}, der skulle gøre robotten mere stabil når den skulle op i høj fart. Det bliver dog ikke brugt, da farten på robotten ikke endte med at være særligt høj.

Derudover er der løbende blevet udført test af motorenes ydeevne, da det er vigtigt at motorerne kører med samme moment

Der er løbende, gennem projektforløbet, blevet testet på udformningen og designet af robotten. Robotten er blevet tilpasset efter hver test. bl.a. kan nævnes længden af robotten der er blev gjort meget kortere. Dette har betydet at robotten nemmere undgår forhindringer, og kan dreje om sin egen akse mellem to forhindringer.

Der er også foretaget tests med lys sensor til robotten, for at verificere at en kage er korrekt samlet op. Tests viste dog store unøjagtighedere og på baggrund af dette blev lys sensoren fravalgt.