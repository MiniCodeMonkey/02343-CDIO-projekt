\chapter{Robot}

\section{Kommunikation}
Til at kommunikere med robotterne har vi valgt online biblioteket \textit{Bluecove}, som er en del af hele \textit{Lejos} biblioteket, hvilket derfor var oplagt at bruge. Desuden var der rigt med eksempler at finde online. 

\section{Styring}
Styringen foregår ved at vi har lavet nogle metoder der der sætter en eller begge motorer i en bestemt tilstand. De forskellige tilstande er at køre med en bestemt hastighed og at stoppe. Metoderne returnerer efter de er blevet kaldt, hvilket vil sige at en motor kører indtil at den bliver stoppet. Det er en meget primitiv måde at styre motorerne, til gengæld har vi fuld kontrol over dem.

Noget om at vi er den eneste gruppe der gør det?

\section{Iterationer}
Det blev hurtigt besluttet i gruppen at det indledende design af en simpel robot, havde meget høj prioritet, da det blev vurderet at det var vigtigt at have en fungerende robot at teste på. Og en vidreudvikling derfor ville tage afsæt i dette design.
Det blev indledningsvis vurderet at lavefødder ville være den bedste fremgangs, da disse ville have et større overfalde areal, og derfor være mindre afhængige af underslaget for at udgå hjulspin ved start og stop. Det var klart, allerede på dette tidspunkt, at denne design beslutning ville resultere i en langsommere robot, men en mere præcis navigering end hjul ville kunne give.
Det viste sig dog at et design med hjul leverede en tilfredsstillende præcision samtidig med en øget hastighed.

\billede{!htbp}{0.4}{BERTA-0_1}{Tidligere udgave af robot design}
\billede{!htbp}{0.4}{BERTA-1_0}{Slut udgave af robot design}

Gennem iterationerne skete der ligeledes en forbedring af selve konstruktionen efterhånden som designet blev forbedret i samarbjede mellem JK og TB. JK havde erfaring med at bygge Lego Technics, hvilket gav en betydelig forbedring af det overordende design, herunder en bedre integration af - og tilgang til NXT blokken.

I den indledende fase af designet, blev der researches en del på nettet, hvor et design til en klo blev fundet, som blev bibeholdt fra start til slut i projektet. Fordelen ved denne klo, er den at den kun drives af en enkelt motor. I andre design ville en motor lukke og åbne kloen, mens en anden ville løfte kloen, og valget af denne klo sparede os derved for at skulle bruge en ekstra NXT blok, da disse kun har 3 RJ-35 tilslutninger for motorer.

Efterhånden som konstruktionen blev gjort mere og mere og robust gennem diverse konstruktionsmæssige forbedringer, gik det op for os, at stivhed ikke nødvendig altid er den bedste fremgangsmåde.
I forbindelse med betjeningen af kloen, identificerede vi et problem der opstod når kloens position skulle bestemmes, altså om den fra starten var åbnet eller lukket. Da denne positionsbestemmelse blev mere mere og mere afgørende efterhånden som konstruktionen blev mere stiv og regid, da motoren i værste fald ville rive konstruktionen fra hinanden, fandt vi ud af, at hvis designet i kraftoverførslen mellem klomotoren og selv kloen blev gjort løs, ville en "`overdrejning"' ikke have nogen betydning. Og som en bonus ville vi så være i stand til med sikkerhed altid at kunne bestemme kloens position ved blot at "`overdreje"' den i 2 sekunder før starten af hver test.

\section{Fra 1 til 2 robotter}
Muligheden for at bruge 2 robotter blev truffet meget tidligt i forløbet. Den endelige beslutning blev først truffet umiddelbart efter det sidste styregruppemøde. Det var netop også her, som Thomas pointerede, at der skulle tages en beslutning om der skulle satses på at lave en enkelt robot og optimere den, eller bruge to. Valget af to robotter blev taget udfra begrundelsen og forventningen at to robotter der arbejdede sammen ville kunne rydde banen hurtigere. Kravene til kommunikation og robotternes indbyrdes forhold ændrede sig dog drastisk. Både fordi kommunikationen var i en Singleton klasse, og fordi det skulle undgås at robotterne tog de samme kager, og kørte ind i hinanden.

\section{RMI}
Da kommunikationen til en robot blev oprettet gennem en Singleton-klasse, løste vi problemet ved at kommunikationen til hver robot fik hver sin process. Der er således noget IPC der foregår mellem vores program og robotterne via RMI.

\billede{!htbp}{0.4}{bluetooth-connection-overview}{Oversigt over NXT flow}

Herover ses hvordan programmet har byttet om på det traditionelle klient-server forhold. Det er gjort for at sikre at programmet og NXT’ernes bluetooth connection kører i seperate processer. Programmet starter en ny process og giver NXT’ens MAC-addresse med som argument. Processen virker som en RMI server, og starter en forbindelse til NXT’en, og lader derefter RMI-klienten kalde metoder fra et interfacet \texttt{IControl} så det er muligt at tilgå de metoder der starter og stopper moterne på NXT’erne.

RMI er den oplagte løsning fordi \rec{-BESKRIV MERE}

\section{Robotternes roller}
For at undgå kollisioner og at robotterne går efter de samme kager, har de fået nogle indbyrdes roller; en er master, og en er slave. Masteren har forkørselsret hvis robotterne kommer for tæt på hinanden, derfor sættes slave-robotten i en yielding-state, hvis den kommer for tæt på masteren. Master-robotten kan derefter navigere uden om slave-robotten når denne er i en yielding-state.

Findes der mange kager på banen bliver den kage tættest på en robot valgt, og robotten begiver sig hen efter den. Når der kun er 2 kager tilbage er de reserveret til masteren. Det er gjort for at undgå at slave-robotten kommer til at køre med den sidste kage og komme for tæt på master-robotten. Den ville således gå i en yielding state, og da der ikke er flere kager ville master-robotten stå og vente til der kommer nye kager.

\section{Test}
Til at teste de metoder som er blevet lavet til at styre robotten har Morten lavet et Testtool til manuel styring af motorerne. Dette har givet god viden i hvordan motererne virker mekanisk, hvordan man ved styrer robotten og hvordan man kommunikerer med robotten. Denne viden er blevet brugt i de metoder vi har lavet i interfacet som deles af RMI. Desuden har det været forsøgt at få implementeret en softstart, der skulle gøre robotten mere stabil når den skulle op i høj fart. Det bliver dog ikke brugt, da farten på robotten ikke er særlig høj.

Derudover er der løbende blevet udført test af motorenes ydeevne, da det er vigtigt at motorerne kører med samme moment

Der er løbende gennem projektforløbet blevet testet på udformningen og designet af robotten, hvor robotten er blevet tilpasset efter hver test. Bl.a. kan nævnes længden af robotten der blev gjort meget kortere efter en test hvor robotten drejede om sin egen akse.