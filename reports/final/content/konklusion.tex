\chapter{Konklusion}
Det er overordnet set lykkedes at opfylde succeskriterierne for projektet. I visse tilfælde vil robotterne dog ramme forhindringer.

Målene for projektet er ikke helt opfyldt. Det er lykkedes at få 2 robotter til at samarbejde omkring løsning af den stillede opgave -- uden at kollidere eller slås om en kage -- men fejlhåndtering er ikke blevet helt gennemarbejdet. Mulighed for at håndtere, at kager/forhindringer bliver flyttet er implementeret og fungerer.\\
I konkurrencen blev det "`kun"' til en 2. plads.

Samarbejdet har fungeret godt igennem hele forløbet. Alle gruppemedlemmer har været engagerede og ydet stort set ligeligt. Strukturen i gruppen har fungeret godt for alle parter.

\section{Konkurrencen}
Det lykkedes ikke at gennemføre nogen baner i konkurrencen uden at måtte genstarte. Dog viste systemet sig at fungere virkelig godt, når de rigtige omstændigheder var til stede.

Eksempelvis har en bug omkring beregning af vinkler betydet, at robotterne i visse tilfælde har drejet flere gange omkring sig selv på vej mod et mål. Dette har kostet tid, men har ikke umiddelbart givet anledning til strafpoint.

En anden problematik har været beregningstiden for stifindingen under visse omstændigheder. Dette har betydet, at der ikke har været en sti tilgængelig for robotten i tide, hvorved robotten blot har fortsat sin igangværende bevægelse. Dette har været problematisk, hvis en robot har været tæt på en forhindring og har kørt mod denne.

Samlet set må det konkluderes, at systemet har potentiale, men at der i nuværende version er nogle uløste problematikker. Disse problematikker vil kunne rettes med en forholdsvis beskeden ekstra indsats.